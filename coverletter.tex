\documentclass[letterpaper]{article}
    \usepackage[margin=1.2in]{geometry}
    %\usepackage{fullpage}
    \usepackage{textcomp}
    \usepackage[T1]{fontenc}
    \usepackage[utf8]{inputenc}
    \usepackage{hyperref}
    \usepackage{ulem}  % underlining
    \usepackage{palatino}

    % insert QR code
    %   https://tex.stackexchange.com/questions/66231/how-to-put-a-picture-to-the-top-right-of-a-page
    \usepackage{graphicx} 
    \usepackage{eso-pic} 



%%%%%%%%%%%%%%%%%%%%%%%%%%%%%%%%%%%%%%%%%%%%%%%%%%%%%%%%%%%%%%%%%%%%%
% MACROS
    \newcommand{\impt}[1]{\uline{#1}}

    % PAGE FORMATTING
    \pagestyle{empty}
    \raggedright
    \setlength{\parskip}{1em}



%%%%%%%%%%%%%%%%%%%%%%%%%%%%%%%%%%%%%%%%%%%%%%%%%%%%%%%%%%%%%%%%%%%%%
% VARIABLES
    
    % PERSONAL DETAILS 
    %    CVname
    %    CVaddresswrap
    %    CVemail
    %    CVphone
    \input _private.tex

    \newcommand{\CVjobTitle}{GIS Analyst}
    \newcommand{\CVcompany}{EDF Renewables}

    % PDF METADATA
    \hypersetup{
        pdfauthor={\CVname},
        pdftitle={\CVname~- \CVjobTitle},
        pdfsubject={\CVjobTitle}
    }

    \hyphenation{
        }



%%%%%%%%%%%%%%%%%%%%%%%%%%%%%%%%%%%%%%%%%%%%%%%%%%%%%%%%%%%%%%%%%%%%%
% DOCUMENT

\begin{document}

\large

\null\hfill June 22, 2020
\vspace{1em}

Dear Hiring Manager:

My extensive GIS and mapping experience working on wide range of projects and
technologies makes me a strong candidate for your \impt{\CVjobTitle} position
at \CVcompany.

With over \impt{nine years of experience}, I have been working with GIS systems
and \impt{collecting field data} in many different types of
projects, listening to customers and clients discussing their needs and
requirements, in order to develop GIS maps, analyses, and solutions that meet
their business goals.  
Technically I have \impt{advanced skills with ArcGIS}
desktop and ArcGIS pro, and have developed plugins and tools with Python.

I live in San Diego and am a legal permanent resident.  

Thank you for considering my application.
I look forward to hearing from you.  


Sincerely,\\
    \hspace{1em} 
    \includegraphics[height=1.2cm]{_ar_sig.png} \\
    \CVname \\
    \small
    \CVaddresswrap \\
    Tel: \CVphone \\
    Email: \CVemail

\end{document}


\iffalse
Role

The GIS Analyst is responsible for creating sophisticated mapping and
visualization tools to help planners, modelers, and decision-makers visualize
real-world events. The GIS Analyst will prepare maps and analytical reports,
maintain and enhance an urban land information system, and design, develop,
test, maintain, and document GIS databases and applications to support
planning, research, and public safety programs. This position is ideal for a
GIS professional with a strong interest in furthering their expertise
performing specialized mapping and analysis in a regional planning environment.

 

Job Responsibilities

• Perform GIS analyses, develop GIS models and databases, design and prepare
maps, and create other GIS-related products.
• Design, develop, integrate, and maintain geographic information databases,
and related tables, forms, and reports used to support regional planning, asset
management, capital improvement projects, and public safety data and
application workflows.
• Support the ongoing development and maintenance of the SPACECORE database and
application tools; edit and code land inventory data in a production
environment.
• Prepare, design, and produce a variety of maps and other graphic
representations displaying layers and attribute data from databases; develop
and maintain data layers using GIS tools and relational databases.
• Perform complex spatial analysis; interpret detailed plans, aerial imagery,
maps and legal descriptions; code and digitize spatial and non-spatial data
into various GIS layers; perform quality control checks to ensure integrity.
• Perform GIS data maintenance, manipulation, analysis, extraction, and
generation; perform data research, investigation, and verification.
• Consult with staff to determine product and information needs and develop
effective, efficient responses to GIS-related requests.
• Develop and implement automated procedures and programs to make routine
geoprocessing workflows faster and more efficient; use object-oriented
programming and relational database technologies to enhance systems and add
functionality to new or existing applications.
• Assist with the ongoing development and maintenance of web sites, desktop,
and mobile applications; review and analyze current applications to identify
opportunities for enhancement.
• Develop visualization applications and tools to enhance effective
decision-making by staff and regional stakeholders.
• Prepare and maintain technical documentation of databases and geoprocessing
programs and procedures; prepare and manage metadata; maintain accurate and
organized records.
• Coordinate with state and local technical committees and agencies to share
GIS expertise; jointly develop GIS databases, applications, and resources.
• Participate in inter-departmental, inter-agency, and binational project teams
assembled for land use, demographic, and econometric modeling, analyses, and
planning.
• Prepare and present written, oral, and visual reports to policy and
stakeholder committees, upper management, member agencies, and working groups.

 

Experience and Qualifications

• A bachelor’s degree with major course work in geography, planning, computer
science, information technology, software engineering, or a related field A
Master’s degree is desirable. A combination of education and recent work
experience may be considered in lieu of a degree
• One to three years of recent career experience in a GIS or a
research-oriented field.
• Demonstrated knowledge of GIS concepts, principles, practices, and
techniques, including cartography, design principles, and map creation
techniques.
• Experience with ESRI’s ArcGIS for Desktop software; experience using ArcGIS
for Server, ArcGIS Pro, Spatial Analyst, 3D Analyst, and Network Analyst
    extensions are desirable.
• Experience designing geodatabases with ArcGIS architecture; experience
compiling and integrating data from multiple GIS sources.
• Demonstrated experience with relational database management systems such as
Microsoft SQL Server; a working knowledge of query design using the SQL
programing language and knowledge of Python is desirable.
• Working knowledge of GIS web services, ArcGIS Online (AGOL), and web mapping
and application development are desirable.
• Working knowledge of emerging GIS technologies for data management, analysis,
and visualization.
• Strong computer skills and proficiency with the Windows operating system and
Microsoft Office applications, especially Word, Excel, and PowerPoint.
• Demonstrated ability to maintain and effectively document GIS databases,
models, programs, and applications.
• Ability to work independently and develop GIS procedures that improve
process, workflow, and analytic capabilities.
• Knowledge of technologies and methods used to create digital orthorectified
imagery and terrain data and basic image processing concepts is desirable.
• Experience working with end-users to determine GIS needs and developing
applications, maps, and other products that best meet those needs.
• Excellent organizational skills, attention to detail, and the ability to
maintain accurate records and work on concurrent projects.
• Strong interpersonal, written, and oral communication skills, including the
ability to effectively communicate technical information to non-technical
audiences.
 
\fi


